\begin{abstract}
    Many coastal regions are heavily impacted by hurricanes and other large storm events, which can drive erosion, 
    inundation, and damage to coastal infrastructure and communities. Measurement of high-energy wave events is difficult 
    and historically there are few measurements during these conditions. As part of DUNEX (DUring Nearshore Event eXperiment)
    in October 2021, arrays of ten to fifty small drifters (microSWIFTs) were deployed using a helicopter and pier-based 
    launcher during a large range of wave conditions. The drifters are outfitted with a GPS unit  and an IMU and record
    position, velocity, accelerations, and rotations that are used to map individual breaking events and circulation. 
    The arrays were designed to enable reconstruction of phase-resolved wave fields on the scales of a few wavelengths 
    (alongshore) and throughout the surf zone (cross-shore).  The wave breaking maps from the drifters are then compared 
    with video imagery to determine the scale of each breaking wave (i.e., crest-length and duration).  These drifter 
    measurements will also be used to validate, force and compare to both phase-averaged (COAWST) and phase-resolved 
    (NHWAVE) wave models. In particular, we plan to use these measurements to further investigate the accuracy of wave 
    breaking parameterizations in phase-averaged models. 
\end{abstract}